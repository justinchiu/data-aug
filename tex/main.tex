% This must be in the first 5 lines to tell arXiv to use pdfLaTeX, which is strongly recommended.
\pdfoutput=1
% In particular, the hyperref package requires pdfLaTeX in order to break URLs across lines.

\documentclass[11pt]{article}

\usepackage[a4paper, total={6in, 8in}]{geometry}
\usepackage{lineno}
\linenumbers

\usepackage{caption,subcaption}

\usepackage{mystyle}

\usetikzlibrary{calc,patterns,angles,quotes}    
\usetikzlibrary{decorations.pathmorphing}
\tikzset{snake it/.style={decorate, decoration=snake}}

\usepackage{natbib}
\bibliographystyle{abbrvnat}
\usepackage{hyperref}

% Standard package includes
\usepackage{times}
\usepackage{latexsym}

% For proper rendering and hyphenation of words containing Latin characters (including in bib files)
\usepackage[T1]{fontenc}
% For Vietnamese characters
% \usepackage[T5]{fontenc}
% See https://www.latex-project.org/help/documentation/encguide.pdf for other character sets

% This assumes your files are encoded as UTF8
\usepackage[utf8]{inputenc}


\title{Learned Data Augmentation and Non-autoregressive Translation}

\author{Justin Chiu \\
  Cornell Tech \\
  \texttt{jtc257@cornell.edu}}

\begin{document}
\maketitle
\begin{abstract}
Abstract
\end{abstract}

\section{Introduction}
Learned models of data are often misspecified.
When the goal of modeling is not density estimation,
but some alternative objective, this misspecification may lead to undesirable
behaviour under the maximum likelihood objective.
In this note, we consider learned data augmentation techniques to edit
each data point so that the data as a whole is more amenable to learning for a particular model.

\section{Problem Setup}
Given data consisting of $(x,y)$ pairs, our goal is to learn a model $q(y \mid x)$
such that it maximizes some objective
\begin{equation}
    \Es{p(x,y)}{\argmax_{\hat{y}}D(q(\hat{y} \mid x), y)},
\end{equation}
where $D$ is some discrepancy measure between our prediction $\hat{y}$ and the true output $y$.

A concrete example of this is translation, where $x$ is a source sentence (for example, German),
$y$ is a target sentence (for example, English),
and $D$ is the BLEU score between our generated translation and the true
reference target sentence.
Our goal is to, given a family of student models $q_\theta(y \mid x)$ indexed by $\theta$,
learn an edit model $q_\phi(\hat{y} \mid y, x)$ whose conditional distribution over $\hat{y}$
is easier for the student model $q_\theta$ to learn.

\section{Method}


\bibliography{bib}

\end{document}
